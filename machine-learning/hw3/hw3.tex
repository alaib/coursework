\documentclass{article}
\newcommand{\BEAS}{\begin{eqnarray*}}
\newcommand{\EEAS}{\end{eqnarray*}}
\newcommand{\BEQ}{\begin{equation}}
\newcommand{\EEQ}{\end{equation}}
\newcommand{\BIT}{\begin{itemize}}
\newcommand{\EIT}{\end{itemize}}

\newcommand{\eg}{{\it e.g.\ }}
\newcommand{\ie}{{\it i.e.\ }}

\newcommand{\ones}{\mathbf 1}
\newcommand{\zeros}{\mathbf 0}
\newcommand{\reals}{{\mbox{\bf R}}}
\newcommand{\integers}{{\mbox{\bf Z}}}
\newcommand{\symm}{{\mbox{\bf S}}}  % symmetric matrices

\newcommand{\nullspace}{{\mathcal N}}
\newcommand{\range}{{\mathcal R}}
\newcommand{\Rank}{\mathop{\bf Rank}}
\newcommand{\Tr}{\mathop{\bf Tr}}

\newcommand{\sign}[1]{\mathop{\textrm{sgn}}(#1)}
\newcommand{\lambdamax}{{\lambda_{\rm max}}}
\newcommand{\lambdamin}{\lambda_{\rm min}}

\newcommand{\EE}{\mathop{\textrm{E}}}
\newcommand{\Cov}{\mathop{\textrm{Cov}}}
\newcommand{\Prob}{\mathop{\bf Prob}}
\newcommand{\Co}{{\mathop {\bf Co}}} % convex hull
\newcommand{\dist}{\mathop{\bf dist{}}}
\newcommand{\argmin}{\mathop{\rm argmin}}
\newcommand{\argmax}{\mathop{\rm argmax}}
\newcommand{\epi}{\mathop{\bf epi}} % epigraph
\newcommand{\Vol}{\mathop{\bf vol}}
\newcommand{\dom}{\mathop{\bf dom}} % domain
\newcommand{\intr}{\mathop{\bf int}}


\newcommand{\nrm}[1]{\left\lVert#1\right\rVert}
\newcommand{\nrmo}[1]{\left\lVert#1\right\rVert_1}
\newcommand{\nrmt}[1]{\left\lVert#1\right\rVert_2}
\newcommand{\nrmnn}[1]{\left\lVert#1\right\rVert_{*}}
\newcommand{\nrmf}[1]{\left\lVert#1\right\rVert_F}

\newcommand{\myexp}[1]{\mathop{\rm exp}\left\{#1\right\}}
\newcommand{\mylog}[1]{\mathop{\rm log}\left\{#1\right\}}
\newcommand{\questions}{\begin{frame}Questions?\end{frame}}
\newcommand{\LL}{\textrm{LL}}
\newcommand{\KL}{\textrm{KL}}
\newcommand{\HH}{\textrm{H}}
\newcommand{\GG}{\textrm{G}}

\newcommand{\Bound}{\textrm{B}}
\newcommand{\bb}{\mathbf{b}}
\newcommand{\aaa}{\mathbf{a}}
\newcommand{\BB}{\mathbf{B}}
\newcommand{\AAA}{\mathbf{A}}
\newcommand{\CC}{\mathbf{C}}
\newcommand{\cc}{\mathbf{c}}
\newcommand{\mm}{\mathbf{m}}
\newcommand{\MM}{\mathbf{M}}
\newcommand{\nn}{\mathrm{\bf neighbors}}
\newcommand{\pa}[1]{{\textrm{\bf pa}}\left(#1\right)}
\newcommand{\pre}[2]{\mathop{\textrm{\bf pnp}}_{#1}\left(#2\right)}
\newcommand{\logsum}{\textrm{logsum}}

\newcommand{\tth}{{\textrm{th}}}
\newcommand{\xx}{\mathbf{x}}
\newcommand{\mmu}{\mathbf{\mu}}
\newcommand{\yy}{\mathbf{y}}
\newcommand{\zz}{\mathbf{z}}
\newcommand{\dd}{\mathbf{d}}
\newcommand{\new}{\textrm{new}}
\newcommand{\old}{\textrm{old}}
\newcommand{\fpr}{\textrm{FPR}}
\newcommand{\tpr}{\textrm{TPR}}
\newcommand{\auc}{\textrm{AUC}}
\newcommand{\yyi}{\yy_i}
\newcommand{\xxi}{\xx_i}
\newcommand{\vvec}[2]{\left[ \begin{array}{c} \mathbf{#1}\\ \mathbf{#2} \end{array}\right]}
\newcommand{\mmat}[4]{\left[ \begin{array}{cc} \mathbf{#1}&\mathbf{#2}\\ \mathbf{#3}&\mathbf{#4} \end{array}\right]}
\newcommand{\xyvec}{\left[ \begin{array}{c} \xx\\\yy \end{array} \right]}
\newcommand{\xyvecc}{\left[ \begin{array}{c} x^1\\y^1 \end{array} \right]}
\newcommand{\eye}{   \left[ \begin{array}{cc} 1 & 0 \\ 0 & 1 \end{array}\right]}
\newcommand{\bket}[2]{\left\langle#1,#2\right\rangle}
\newcommand{\bbket}[2]{\left\llangle#1,#2\right\rrangle}
\newcommand{\redq}{\textcolor{red}{q}}
\newcommand{\blup}{\textcolor{blue}{p}}
\newcommand{\BIEA}{\begin{IEEEeqnarray*}}
\newcommand{\EIEA}{\end{IEEEeqnarray*}}
\newcommand{\BIEAN}{\begin{IEEEeqnarray}}
\newcommand{\EIEAN}{\end{IEEEeqnarray}}
\newcommand{\pmin}{\mathop{\textrm{minimize}}}
\newcommand{\psubjto}{\textrm{subject to}}
\newcommand{\WW}{\mathbf{W}}
\newcommand{\ww}{\mathbf{w}}
\newcommand{\YY}{\mathbf{Y}}
\newcommand{\XX}{\mathbf{X}}
\newcommand{\UU}{\mathbf{U}}
\newcommand{\uu}{\mathbf{u}}
\newcommand{\VV}{\mathbf{V}}
\newcommand{\vv}{\mathbf{v}}
\newcommand{\PP}{\mathbf{P}}
\newcommand{\pp}{\mathbf{p}}
\newcommand{\rr}{\mathbf{r}}
\newcommand{\RR}{\mathbf{R}}
\newcommand{\ee}{\mathbf{e}}
\newcommand{\II}{\mathbf{I}}
\newcommand{\DD}{\mathbf{D}}

\newcommand{\aalpha}{{\boldsymbol\alpha}}
\newcommand{\llambda}{{\boldsymbol\lambda}}
\newcommand{\ddelta}{{\boldsymbol\delta}}
\newcommand{\otherwise}{\textrm{otherwise}}
\newcommand{\answer}{\fbox{\tt answer} }
\newcommand{\abs}[1]{\left| #1 \right|}

\newcounter{problemCtr}
\newcommand{\newproblem}[1]{\hrule\paragraph{Problem \theproblemCtr (#1)}\stepcounter{problemCtr}}


\newcounter{HW}

\usepackage{amsthm}
\usepackage{graphicx}
\usepackage{natbib}
\usepackage{algorithm}
\usepackage{algorithmic}
\usepackage{amsmath}
\usepackage{hyperref}
\usepackage{tikz}


\newtheorem{remark}{Remark}
\newtheorem{lemma}{Lemma}
\newtheorem{definition}{Definition}
\newtheorem{proposition}{Proposition}
\newtheorem{assumption}{Assumption}
\newtheorem{corollary}{Corollary}
\newtheorem{theorem}{Theorem}


\begin{document}
\author{Wile E. Coyote}
\setcounter{HW}{3}
\title{COMP  790-124, HW\theHW}
\maketitle

{ Deadline: 11/9/2012 11:59PM EST}

{ Submit \texttt{hw\theHW.pdf} by e-mail,  \url{mailto:vjojic+comp790+hw\theHW@cs.unc.edu}}.


\noindent\rule{\textwidth}{3pt}
 We will assume existence of an $N$ long backbone sequence $s$ in an $N$. In this assignment the alphabet will be of size 4, corresponding to nucleotides. We will construct a hidden Markov model that generates a shorter sequence from the backbone sequence. The shorter sequence will consist of two parts of equal length $L$. First part of sequence corresponds to offsets, 1 through $L$, and the second part of the sequence corresponds to offsets, $L+1$ to $2L$. With each offset $i$ in the two-part sequence we will associate a hidden variable that points to a position in the backbone sequence.
 The probability of a letter $x_i$, given pointer $h_i$ is
 \[
 p(x_i|h_i) = \begin{cases}
 0.99,& \mbox{if } x_i = s_{h_i} \\
 \frac{0.01}{a - 1},& \mbox{if } x_i \neq s_{h_i}. \\
 \end{cases}
 \]
 Finally, we define transition probability on the $h_i$
 \[
h_{i+1}|h_i \propto
\begin{cases}
\textrm{TruncPoiss}(h_{i+1} - h_i), & \mbox{if } i = L\\
\pi_{\textrm{ins}}, &\mbox{if } h_{i+1} = h_i, i \neq L \\
\pi_{\textrm{del}}, & \mbox{if } h_{i+1} = h_i+2, i \neq L, h_i+2 \leq N \\
\pi_{\textrm{copy}}, &  \mbox{if } h_{i+1} = h_i+1,i \neq L, h_i+1 \leq N.
\end{cases}
\]
where TruncPoiss denotes a truncated poisson, given by
\[
\textrm{TruncPoiss}(l) \propto \begin{cases}
\textrm{Poiss}(l,\lambda=100), &\mbox{if } 90\leq l \leq 110 \\
0 , &\mbox{otherwise }
\end{cases}
\]
 Hence, we move the pointer forward with an occasional skip (deletion) or lag (insert), except when we reach offset $L$ where we make a leap of approximately 100 positions.

We will use a {\tt logsum} function
\begin{verbatim}
function s = logsum(vec)
m = max(vec);
s = log(sum(exp(vec - m))) + m;
\end{verbatim}
and {\tt logProbPoiss}
\begin{verbatim}
function logProb = logProbTruncPoiss(i,lambda)
logProb = log(lambda)*i + (-lambda) - sum(log(1:i));
\end{verbatim}
\pgfdeclarelayer{background}
\pgfdeclarelayer{foreground}
\pgfsetlayers{background,main,foreground}
\begin{center}
\begin{tikzpicture}[->,shorten >=0pt,auto,node distance=1.6cm,semithick]

\tikzstyle{var}=[minimum size=.7cm,circle,fill=white,draw=black,text=black]
\tikzstyle{ovar}=[minimum size=.7cm,circle,fill=gray,draw=black,text=black]



\node[var] (h1) {$h_1$};

\node[ovar] (x1) [below of=h1] {$x_1$};


\node[var] (h2) [right of=h1] {$h_2$};
\node[ovar] (x2) [below of=h2] {$x_2$};


\node[var] (h3) [right of=h2] {$h_3$};
\node[ovar] (x3) [below of=h3] {$x_3$};


\node[var] (h4) [right of=h3] {$h_4$};
\node[ovar] (x4) [below of=h4] {$x_4$};

\begin{pgfonlayer}{background}


\path (h1) edge (x1);
\path (h2) edge (x2);
\path (h3) edge (x3);
\path (h4) edge (x4);





\path (h1) edge (h2);
\path (h2) edge (h3);
\path (h3) edge (h4);


\end{pgfonlayer}
\end{tikzpicture}
\end{center}


\newproblem{3pt} Given the HMM model specification above implement a forward pass in the HMM.
 Storing the transition matrix explicitly would be too costly. But the matrix is very sparse. With the exception of $i=L$, from a given offset $h_i$ we can transition to {\em at most} three different offsets. Hence computation of the forward pass will require us to explicitly account for these possibilities. Note that if $h_i=N$ then the only possible transition is to $h_{i+1}=N$.

Note that transition from $h_{L}$ to $h_{L+1}$ is made according to a Poisson.
\begin{verbatim}
function m_f = fw(s,x,pins,pdel,pcopy)
N = length(s);
L = length(x)/2;
m_f = -realmax*ones(N,2*L);

logpins = log(pins);
logpdel = log(pdel);
logpcopy = log(pcopy);
for a=1:4
    for b=1:4
        logmut(a,b) = log(0.99)*(a==b) + log(0.01)*(a~=b);
    end
end

m_f(1:N,1) = -log(N) + logmut(s,x(1));
for i=2:2*L
    for prev=1:N
        if i~=L+1
            % insert
            vala = m_f(prev,i);
            valb = m_f(prev,i-1)+ logmut(s(prev),x(i)) + logpins;
            m_f(prev,i) = logsum([vala valb]);
            if prev<=N-2
                % delete
                vala = m_f(prev+2,i);
                valb = ...
                m_f(prev+2,i) = logsum([vala valb]);
            end
            if prev<=N-1
                % copy
                vala = m_f(prev+1,i);
                valb = ...
                m_f(prev+1,i) = logsum([vala valb]);
            end
        else
            if prev+90<=N
                for next=prev+90:min(prev+110,N)
                    vala = m_f(next,i);
                    valb = ...;
                    m_f(next,i) = logsum([vala valb]);
                end
            end
        end
    end
end
\end{verbatim}
Run forward pass on inputs stored in {\tt hw\theHW.mat} and run this script
\begin{verbatim}
m_f = fw(seq,x,0.005,0.005,0.99);
logProb = logsum(m_f(:,end))
\end{verbatim}
The resulting {\tt logProb} is \answer

\newproblem{3pt} Implement a backward pass.

\begin{verbatim}
function m_b = bw(s,x,pins,pdel,pcopy)
N = length(s);
L = length(x)/2;
m_b = -realmax*ones(N,2*L);

logpins = log(pins);
logpdel = log(pdel);
logpcopy = log(pcopy);
for a=1:4
    for b=1:4
        logmut(a,b) = log(0.99)*(a==b) + log(0.01)*(a~=b);
    end
end

m_b(:,2*L) = 0;
for i=2*L-1:-1:1
    i
    for next=1:N
        if i~=L
            % insert
            vala = m_b(next,i);
            valb = m_b(next,i+1)+ logmut(s(next),x(i+1)) + logpins;
            m_b(next,i) = logsum([vala valb]);
            if next-2>=1
                % delete
                vala = m_b(next-2,i);
                valb = ...
                m_b(next-2,i) = logsum([vala valb]);
            end
            if next-1>=1
                % copy
                vala = m_b(next-1,i);
                valb = ...
                m_b(next-1,i) = logsum([vala valb]);
            end
        else
            if next-90>=1
                for prev=max(1,next-110):next-90
                    vala = m_b(prev,i);
                    valb = ...;
                    m_b(prev,i) = logsum([vala valb]);
                end
            end
        end
    end
end
\end{verbatim}
To check your implementation run following code
\begin{verbatim}
m_f = fw(seq,x,0.005,0.005,0.99);
m_b = bw(seq,x,0.005,0.005,0.99);
mm = m_f + m_b;
logsum(mm(:,1))
logsum(mm(:,end))
\end{verbatim}
If the two logsum calls output different values, you have a bug.

\newproblem{3pt} Implement Viterbi forward and backward pass. To do this transform your forward pass and backward by replacing the {\tt logsum} with {\tt max}

\newproblem{2pt} Use MATLAB commands {\tt tic} and {\tt toc} to measure the time that it takes to run standard forward pass and Viterbi forward pass to complete. The ratio of the these times is \answer. Is one of these functions faster? If so, why?

Hint: One way to answer this question is to use MATLAB's profiler. Before calling a function you want to profile do folowing
\begin{verbatim}
profile clear
profile on
\end{verbatim}
Run your code and once done (or once you interrupt it)
\begin{verbatim}
profile viewer
\end{verbatim}
Rest ought to be self-explanatory.



\newproblem{3pt} Modify your Viterbi forward pass to also record which of the states  $h_{i-1}$ was the most likely to have given rise to the current state $h_{i}$. This is sometimes called trace or backward pointers.

Implement code that starting with the state $h_L$ with highest probability in the forward pass, backtracks according to the trace recording the path. Apply this procedure to long sequence {\tt seq} and short sequence {\tt x} in {\tt hw\theHW.mat}.

Paste most likely sequence of offsets in {\tt seq} used to generate {\tt x}
\begin{verbatim}
...
\end{verbatim}



\end{document}
